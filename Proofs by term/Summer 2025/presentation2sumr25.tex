% !TEX TS-program = pdflatex
% !TEX encoding = UTF-8 Unicode

% This file is a template using the "beamer" package to create slides for a talk or presentation
% - Talk at a conference/colloquium.
% - Talk length is about 20min.
% - Style is ornate.

% MODIFIED by Jonathan Kew, 2008-07-06
% The header comments and encoding in this file were modified for inclusion with TeXworks.
% The content is otherwise unchanged from the original distributed with the beamer package.

\documentclass{beamer}


% Copyright 2004 by Till Tantau <tantau@users.sourceforge.net>.
%
% In principle, this file can be redistributed and/or modified under
% the terms of the GNU General Public License, version 2.
%
% However, this file is supposed to be a template to be modified
% for your own needs. For this reason, if you use this file as a
% template and not specifically distribute it as part of a another
% package/program, I grant the extra permission to freely copy and
% modify this file as you see fit and even to delete this copyright
% notice. 


\mode<presentation>
{
  \usetheme{Antibes}
  % or ...

  \setbeamercovered{transparent}
  % or whatever (possibly just delete it)
  
  % (Removed duplicate \documentclass and preamble packages)
  % % If you want to keep color definitions and beamer color settings, move them outside this block.

  % TO AUTOCOMMENT OUT, Ctrl-/slash
  % \definecolor{mqred}{RGB}{166, 25, 46}
  % \definecolor{mqdeepred}{RGB}{118, 35, 47} blue
  % \definecolor{mqgray}{RGB}{55, 58, 54}
  % \definecolor{mqlightgray}{RGB}{237, 235, 229}
  % \definecolor{mqmagenta}{RGB}{198, 0, 126}
  % \definecolor{dark-green}{RGB}{0, 100, 0}
  % \usecolortheme[named=mqred]{structure}
  % \setbeamercolor{title in head/foot}{bg=mqlightgray, fg=mqgray}
  % \setbeamercolor{author in head/foot}{bg=mqdeepred}
\definecolor{myblue}{RGB}{30, 60, 114}
\definecolor{mydeepblue}{RGB}{15, 32, 65}
\definecolor{mygray}{RGB}{55, 58, 54}
\definecolor{mylightgray}{RGB}{237, 235, 229}
\definecolor{mycyan}{RGB}{0, 180, 216}
\definecolor{dark-green}{RGB}{0, 100, 0}
\usecolortheme[named=myblue]{structure}
\setbeamercolor{title in head/foot}{bg=mylightgray, fg=mygray}
\setbeamercolor{author in head/foot}{bg=mydeepblue}
\setbeamercolor{page number in head/foot}{bg=mydeepblue, fg=mylightgray}
}

\usepackage[english]{babel}
% or whatever

\usepackage[utf8]{inputenc}
% or whatever

\usepackage{times}
\usepackage[T1]{fontenc}
% Or whatever. Note that the encoding and the font should match. If T1
% does not look nice, try deleting the line with the fontenc.


\title[A Discussion on Symmetry and Awesome Sets] % (optional, use only with long paper titles)
{A Discussion on Symmetry and Awesome Sets}

\subtitle
{An Introduction to the Isometries of \texorpdfstring{$\mathbb{R}^3$}{R3}}

\author[Ryan Baldwin] % (optional, use only with lots of authors)
{R. Baldwin}
% - Give the names in the same order as the appear in the paper.
% - Use the \inst{?} command only if the authors have different
%   affiliation.

\institute[Swarthmore College] % (optional, but mostly needed)
{
  \inst{}%
  Math/Stat Department\\
  Swarthmore College
}
\date[CFP 2025] % (optional, should be abbreviation of conference name)
{Swarthmore College}
% - Either use conference name or its abbreviation.
% - Not really informative to the audience, more for people (including
%   yourself) who are reading the slides online

\subject{Math}
% This is only inserted into the PDF information catalog. Can be left
% out. 

% Delete this, if you do not want the table of contents to pop up at
% the beginning of each subsection:
\AtBeginSubsection[]
{
  \begin{frame}<beamer>{Outline}
    \tableofcontents[currentsection,currentsubsection]
  \end{frame}
}


% If you wish to uncover everything in a step-wise fashion, uncomment
% the following command: 

%\beamerdefaultoverlayspecification{<+->}


\begin{document}

\begin{frame}
  \titlepage
\end{frame}

\begin{frame}{Outline}
  \tableofcontents
  % You might wish to add the option [pausesections]
\end{frame}

\section{Motivation}

\subsection{A Reminder on the Definition}

\begin{frame}
  \frametitle{What are $aba$-sets?}
  \framesubtitle{Also known as "awesome"-sets, A-sets}

  \begin{alertblock}{Definition}
     Let $G$ be a group. \newline
     A set $S \subseteq G$ is an $aba$-$set$ if for all $a,b \in S$, $aba \in S$.
  \end{alertblock}

  \begin{itemize}
  \item Previous results:
  \begin{itemize}
  
  \pause  
  \item If $x \in S$ is an element of odd order, then $\langle x \rangle \in S$.
  \end{itemize}
  \end{itemize}
\end{frame}

\begin{frame}{How much can these sets generate?}
  Questions to consider:

  \begin{itemize}
  \pause
  \item Can an $aba$-set with only generators and the identity recover the whole group?
  \pause
  \item Is there a more geometric quality that $aba$-sets induce?\newline
  \end{itemize}

  \pause
  Let's looks at an example to further explore this!
\end{frame}

\subsection{Motivating Example}

\begin{frame}{The Dihedral Group \hfill \makebox[216pt][r]{$\forall a,b\in S,\ aba \in S$}}
  \begin{example}
    Let $S = \{1, \sigma, \zeta\} \subset D_n$ be a set with property $aba$-closure. What must follow is that $S$ must contain ('generate') all powers of the flips and rotations.
  \end{example}

  \pause
  \begin{proof}
    \begin{itemize}
      \pause
      \item Flips have order 2, meaning $\sigma = \sigma^{-1} \in S$ a priori.
      \pause
      \item We can \alert{build-up} all powers of $zeta$ through different recursive combinations with the identity.\newline 
      For example, $\zeta 1 \zeta = \zeta^2$ and $\zeta \zeta \zeta = \zeta^3$ and so forth.
      \item Thus, we can generate all powers of the flip-$\sigma$ and rotation-$\zeta$ elements.
    \end{itemize}
  \end{proof}
\end{frame}

\section{Discussion}

\subsection{\texorpdfstring{$D_n$}{Dn} and Isometries of \texorpdfstring{$\mathbb{R}^3$}{R3}}
\begin{frame}{Subgroups of $\mathbb{R}^3$ \hfill \makebox[238pt][r]{$\forall a,b\in S,\ aba \in S$}}
  To find groups similar to the dihedral group, let's impose firm positions on the vertices of $D_n$ on a plane.\newline

  \pause
  In other words, let us look at what \alert{space} $D_n$ lives in!

\end{frame}


\begin{frame}{Subgroups of $\mathbb{R}^3$ \hfill \makebox[238pt][r]{$\forall a,b\in S,\ aba \in S$}}
  \begin{itemize}
    \pause
    \item The dihedral groups are all $subgroups$ of the group of rotations in $\mathbb{R}^3$ (SO3).
    \pause
    \item Other examples of finite subgroups of SO3 are" \textcolor{dark-green}{cyclic, tetrahedral, octohedral group}.\newline
  \end{itemize}

\end{frame}

\begin{frame}{Another Interesting Example \hfill \makebox[184pt][r]{$\forall a,b\in S,\ aba \in S$}}
\setbeamercolor{block body}{bg=dark-green!10}
\setbeamercolor{block title}{bg=dark-green, fg=white}
  \begin{block}{Remark}
    Recall that if $|x|$ is odd and in $S$, $\langle x \rangle \in S$. 
    \pause
    We can extend this to \textit{any} cyclic subgroup, so long as we start with the identity! That is to say:\newline

    Let $x \in G$ have finite order. If $\{1, x\} \in S$, then $\langle x \rangle \in S$.\newline 
    
    Furthermore, if $G$ is a cyclic group, then the entire group is an $aba$-\textit{group!}
  \end{block}
\end{frame}
\begin{frame}{Make Titles Informative.}
\end{frame}

\begin{frame}{Make Titles Informative.}
\end{frame}

\begin{frame}{Make Titles Informative.}
\end{frame}


\subsection{Ideas Moving Forward}

\begin{frame}{Groups to Inspect}
  \begin{itemize}
    \item Focus on investigating the $aba$-sets of subgroups and semi-direct products of primarily $\mathbb{R}^3$ (a topic for next week).
    \pause
    \item Observe if there is any correlation between said $aba$ sets (specifically with sets only containing the generators and identity of these groups).

  \end{itemize}
\end{frame}


\begin{frame}{Programming}
  \begin{itemize}
    \item Working on code that will speed up the process of finding $aba$-sets.


    \item *Potentially attempt to create a 3D program that may enumerate any \textit{geometric} qualities of $aba$-sets.


  \end{itemize}
\end{frame}


\section*{Concluding Remarks}

\begin{frame}{Summary}

  % Keep the summary *very short*.
  \begin{itemize}
  \item
    I am now not only investigating $aba$-sets from an algebraic viewpoint, but seeing if the \alert{space} where a group lives in induces a constraint in algebraic results.\newline
 

  \pause
  \item
    Observing how $aba$-sets act on symmetries may lead into questions like how much of an original group is guaranteed to be \alert{encoded} in an $aba$-set.\newline

  \pause

  \item
    While larger conjectures in $\mathbb{R}^n$ would be awesome, I will likely continue to work with grups in $\mathbb{R}^2$ and $\mathbb{R}^3$ due to some computational concerns (things get \textit{big} and confusing).


    
  \end{itemize}
  
\end{frame}

\begin{frame}
  
  \begin{center}
    {\Huge Thank you!}
  \end{center}

  

\end{frame}

\end{document}


