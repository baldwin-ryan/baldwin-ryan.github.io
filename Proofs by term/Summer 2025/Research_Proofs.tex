% !TEX TS-program = pdflatex
% !TEX encoding = UTF-8 Unicode

\documentclass[a4paper,12pt]{article}

% Include necessary packages
\usepackage{amsmath, amssymb, amsthm}
\usepackage{tcolorbox}
\usepackage{physics}
\usepackage{hyperref}
\usepackage{enumitem}
\usepackage{geometry}
\geometry{margin=1in}

% Define custom tcolorbox environment (without numbering)
\newtcolorbox{problem}[2][]{colback=red!5!white,colframe=red!75!black,
title=Problems: #2,#1}

% Define custom commands
\newcommand{\im}{\mathrm{i}}

%%% packages that will follow me to the end of time
\usepackage{graphicx} % Required for inserting images
\usepackage{amsmath}
\usepackage{amssymb}
\usepackage{pifont}
\usepackage{enumitem}
\usepackage{scalerel}[2016-12-29]
\usepackage{amsfonts}
\usepackage{graphicx}
\usepackage{pifont}
\usepackage{amssymb}

\newcommand{\N}{\mathbb{N}}
\newcommand{\Z}{\mathbb{Z}}
\newcommand{\Q}{\mathbb{Q}}
\newcommand{\R}{\mathbb{R}}
\newcommand{\C}{\mathbb{C}}

\setlength{\parindent}{20pt}  % however much indentation is want

\setlength{\parskip}{1em}  % value between paragraphs;  any value like 0.5em, 10pt, etc.

\begin{document}

%alt + z for text-wrap around

%============================================================================================================================
%============================================================================================================================

\begin{Large}
    \textsf{\textbf{Proof Collection 500 Goal}}
\end{Large}

\textsf{\textbf{Mentee:}} Ryan Baldwin, \href{mailto:rbaldwi2@swarthmore.edu}{\text{rbaldwi2@swarthmore.edu}}

\textsf{\textbf{Mentor:}} \text{Dr. Devlin}
\vspace{2ex}

This is a collection of proofs I have done over the summer of 2025 as a student researcher.

\newpage %====================================================================================================


\textit{Defn:} Let G be a group. A set $S \subseteq G$ is "awesome" iff $\forall a , b \in S$, $aba \in S.$

0. If G is a group, a set S is "awesome" iff for all a and b in S, we have that aba is in S as well.

\textit{Note:} For proofs 1-3, the group operation is written multiplicatively for simplicity.

------------

1. If H is a subgroup of G, then H is awesome.

\begin{proof}
    Since $H$ is a subgroup of $G$, $H$ is closed under the group operation, a priori. Let $a$ and $b$ be any element in $H$. Then the element $aba$ is an element of $H$ for any $a , b \in H.$
\end{proof}

2. If G is abelian and if every element of G has order 2, then \textit{every} subset of G is awesome.

\begin{proof}
    Let $S$ be a finite set. Then let the following map define the awesome property:

    \begin{align*}
        S \times S & \rightarrow S \\
        (a, b) &\mapsto aba, \quad \forall a, b \in S
    \end{align*}

    We prove this statement by induction and construct a base case. First, we consider a subset of $G$ with one element denoted $S_1 := \{a_1\}.$ Since every element of $G$ has order $2$, every element of $G$ is its own inverse. Hence,  $(a_1, a_1) = a_1 \in S_1$. 
    
    Let us append another distinct element of $a_2 \in G$, and define $S_2 := S_1 \cup \{a_2 \}= \{a_1\} \cup \{a_2\}$. Because  every element of $G$ has order $2$ and the group operation is abelian, we prove this subset is awesome. We see that $(a_1, a_2) = a_1 a_2 a_1 = a_1 a_1 a_2 = (a_1)^2 a_2 = 1 a_2 = a_2 \in S_2$. Furthermore, we notice that $(a_2, a_1) = a_2 a_1 a_2 = (a_2)^2 a_1 = a_1 \in S_2$.
    
    Finally, to conclude our base case we consider the case  where we append another unique element, we denote $a_3 \in G$ to the subset $S_2$ and define $S_3 := S_2 \cup \{a_3\}.$ Similar to the subsets $S_1$ and $S_2$, the subset $S_3$ must also be awesome. This is because the awesome set's law of composition is abelian and every element will always have an order of $2$, these two facts will have the awesome property equaling $aba = (aa)b = 1b =b$ for all $a, b$ in $S$. 

    We induct on the number of elements in the subset, and we assume that this claim holds true for $n$ number of unique elements in $G$. We will show that this claim holds true for $n+1$ elements of $G$, thus proving the claim.

    Let $S_n$ be a subset of $G$ such that it is awesome, and let $a_{n+1}$ be an element of $G$ that is not in $S_n$. We define $S_{n+1} := S_n \cup \{a_{n+1}\}$, and we aim to show that this is remains an awesome subset of $G$. We let $a_i$ be any element of $S_n$ for $1 \leq i \leq n$ and utilize the fact that any such element in the set has an order of $2$. Furthermore, we recall that given $a_{n+1}$ is an element of $G$, it must also have an order of $2$. Now we show that the set $S_{n+1}$ is awesome directly. We first notice that for any $a_i \in S_n$ the awesome property $(a_i, a_{n+1}) = (a_i) a_{n+1} (a_i) = (a_i)^2 a_{n+1} = a_{n+1} \ S_{n+1}$ will always be true. Furthermore, we see that for any $a_i \in S_n$, the property $(a_{n+1}, a_i) = (a_{n+1})^2 a_i = a_i \in S_{n+1}$ will also always be true. Thus, it is true then that the subset $S_{n+1}$ is an  awesome set. Therefore, any subset of $G$ with the aforementioned properties will always be an awesome set.

\end{proof}

Proof 2:
\begin{proof}
Let $S$ be an arbitrary subset of $G$.  (Goal to prove for all $a,b \in S$ we have $(a,b) \in S$)

[Finish writing this direct proof]



\end{proof}

3. Suppose G is the dihedral group with 10 elements.  Suppose S contains the flip and the rotate elements.  If S is awesome, show that S must contain all the elements of the group.

$S$ must contain $ \{1=\sigma^2=\zeta^5, \sigma, \zeta, \zeta^2, \zeta^3, \zeta^4\}.$  [You proved that already]

Also!  The set $ \{1=\sigma^2=\zeta^5, \sigma, \zeta, \zeta^2, \zeta^3, \zeta^4\}$ happens to be awesome already (without needing to add anything else).

\begin{proof}
    The dihedral group with $10$ elements is $D_5 := \langle \sigma, \zeta | \sigma^2 = \zeta^5 = 1, \zeta\sigma = \sigma\zeta^4 \rangle$. Assuming that the set $S \supseteq \{\sigma, \zeta\}$ is awesome, we will show that the set must contain all powers of the generators of $D_5$, which will consequently prove that $S = D_5$.

    Since $S$ is an awesome set, $\sigma \zeta \sigma = \sigma (\zeta \sigma) = \sigma^2 \zeta ^4 = \zeta^4$ is in $S$. Consequently, what must follow is that $\zeta^4 \zeta^4 \zeta^4 = \zeta^2$ must also be an element of $S$. We also note that $\zeta^3$ must also be in the set as $\sigma\zeta^2 \sigma = \sigma \zeta (\zeta \sigma) = \sigma \zeta\sigma \zeta^4 = \zeta^4 \zeta^4 = \zeta^3$. And because $\zeta^2$ and $\zeta$ must be elements of $S$, it follows that $S$ must contain the identity element $1$ as follows: $\zeta^2 \zeta \zeta^2 = 1$. Thus, $S = \{1=\sigma^2=\zeta^5, \sigma, \zeta, \zeta^2, \zeta^3, \zeta^4\}.$

    Given that the generators of $D_5$ and the identity must be in $S$, what follows is that the remaining group elements can be generated through recursive composition. This is because if $a, b$, and $1$ are all elements of an awesome set $S$, then $(1a1)(1b1) = (1) ab 1$ is an element of $S$. Indeed, what then follows is that $\sigma \zeta, \sigma \zeta^2, \sigma \zeta^3, \sigma \zeta^4$ must be elements of $S$. Thus, $S = \{1, \sigma, \zeta, \zeta^2, \zeta^3, \zeta^4, \sigma \zeta, \sigma \zeta^2, \sigma \zeta^3, \sigma \zeta^4\}$ and is therefore the whole group

\end{proof}

4. Suppose $A_n$ is a sequence such that $A_0 = 1$ and that for all $n > 0$, we have $A_n = 2A_{n-1}$. Use induction to prove that for all $n\geq0$ we have $A_n = 2^n$.

\begin{proof}
    To begin the base case, we note that for the initial sequence $A_0 = 1 = 2^0$. And since the sequence follows the recurssion $A_n = 2A_{n-1}$ for all $n > 0$, for $n = 1$ we have $A_1 = 2 A_0 =2^1$. Furthermore, for $n =2$ we see that $A_2 = 2 A_1 = 2^2 A_0 = 2^2$. And in a similar fashion to the previous cases, for $n = 3$ what follows is that $A_3 = 2 A_2 = 2 (2^2 A_0) = 2^3 A_0 = 2^3$. 

    We induct on the sequence and assume that this relation is true for the $n-1$ sequence. That is, we assume $A_{n-1} = 2^{n-1}$. Because of this assumption, we see that $A_n = 2 A_{n-1} = 2 2^{n-1}= 2^n$. Thus, we have shown that for all $n \geq 0$, $A_n = 2^n$. Therefore, the claim is true.

\end{proof}


\newpage
--------------------------------------------------------------------------------

Tasks from June 6


Redo: Finish writing the proofs of 2 and 3 (restated version of 3). ****

5. Prove that if $G = D_n$ (dihedral group with $2n$ elements) and if $S = \{\sigma, 1, \zeta, \zeta^2 , \zeta^3 , \ldots , \zeta^{n-1}\}$, then $S$ is awesome (????) *****


5.a Is this true: $\zeta \sigma = \sigma \zeta^{n-1} = \sigma \zeta^{-1}$

\begin{proof}
    We shall prove this relation to be true by showing that their resulting  clockwise-order sets are equal. Suppose we have an $n$-gon in a two-dimensional plane. We define $\sigma$ to represent flips based off of a vertix with a position we label $v_1$. Furthermore, let $v_1$ be a vertix that is part of an ordered set of unique clockwise-oriented vertices denoted by  $\{v_1, v_2, v_3, v_4, \ldots, v_n\}$. Since the $n$-gon is spaced on a two-dimensional plane, there can only be two flips centered on an such vertix $v_i$. Hence, $\sigma ^2 = 1$, the identity state. Furthermore, we define $\zeta$ to represent a clockwise rotation where all vertices replace the next proceeding vertix in clockwise recession. That is, $v_1 \mapsto v_2, v_2 \mapsto v_3, v_3 \mapsto v_4, \ldots, v_{n-1} \mapsto v_n,$ and $v_n \mapsto v_1$. Because the vertices of the $n$-sided polygon are unique, there are only $n$ unique clockwise rotations. This then implies that $\zeta ^ n = 1$. 

    If it can be shown that $v_2 \mapsto v_1$ in a clockwise-orientation for each relation, then the resulting clockwise-order sets are equal and thus the relations are equivalent. Let us start with a flip $\sigma$ centered on $v_1$ (that is, $v_1$ is fixed). After doing this, we rotate the $n$-gon clockwise by one vertix (i.e. performing $\zeta$) and have $v_2$ take the original position of $v_1$. In a clockwise rotation, $v_2 \mapsto v_1$ with $v_2$ in the original position of $v_1$ . We recall that because all vertices are ordered uniquely, the $n$-gon must have a clockwise rotation on the given set $\{v_2, v_1, v_n, v_{n-1}, \ldots, v_3\}$.

    Now let us start with an inverse rotation by one vertix (denoted by $\zeta ^{-1}$) centered on $v_1$. It follows that $v_2$ must then take the original position of $v_1$, and we have  $v_1 \mapsto v_2$ clockwise-facing,  where $v_2$ is in the position of what used to be $v_1$.
    We then compose a flip centered on the \textit{position} of what used to be $v_1$. We see that $v_2 \mapsto v_1$ must then be true, and given $v_2$ remains in the original position of $v_1$, we obtain the following set that defines where every vertix will map to when composed with a clockwise rotation: $\{v_2, v_1, v_n, v_{n-1}, \ldots, v_3\}$. Given that these two sets are equivalent, it follows that $\zeta \sigma = \sigma \zeta^{-1}$. And since one couter-clockwise rotation is the same as $n-1$ clockwise rotations, we know that $\zeta^{-1} = \zeta^{n-1}$. Therefore, $\zeta \sigma = \sigma \zeta^{n-1} = \sigma \zeta^{-1}$ for any $n$-gon.



\end{proof}

5.b What's $\zeta^a \sigma \zeta^a$?

\begin{proof}
    We prove this by induction. By problem 5.a, we know that $\zeta \sigma = \sigma \zeta^{n-1}$ for all $n \geq 3$. We consider the base case $ \zeta \sigma \zeta$, where we set  $\zeta^3 =1$. We see that $(\zeta \sigma) \zeta = (\sigma \zeta^{3-1}) \zeta = \sigma \zeta^3 = \sigma$.

    We induct on the exponent of $\zeta$ and assume this relation holds up to $m$. W shall prove this relation holds for $m+1$ and therefore for all $m \geq 3$. We notice that $\zeta^{m+1} \sigma \zeta^{m+1} = \zeta^1 (\zeta^m \sigma \zeta^m) \zeta^1 = \zeta^1 \sigma \zeta^1 = \sigma$. Therefore, $\zeta^m \sigma \zeta^m = \sigma$ for all $m \geq 3$.

\end{proof}
 
5.c What's $\sigma \zeta^a \sigma$?

\begin{proof}
    We prove this by induction on the exponent of $\zeta$. Considering the base case $\sigma \zeta \sigma$ with the determined relation from problem 5.a, we see that $\sigma \zeta \sigma = \sigma (\zeta \sigma) = \sigma \sigma \zeta^{n-1} = \zeta^{1(n-1)}$ for all $n \geq 3$. 

    Assume this holds for all exponents of $\zeta$ up to $m$ such that $\sigma \zeta^m \sigma = \zeta^{m(n-1)}$ for all $n \geq 3$. We shall prove that this hold for the $m+1$ exponent, and thus all exponents of $\zeta$. We see that $\sigma \zeta^{m+1} \sigma = \sigma \zeta^m (\zeta \sigma) = \sigma \zeta^m \sigma \zeta^{n-1}$. We employ our inductive hypothesis and observe that $(\sigma \zeta^m \sigma) \zeta^{n-1} = \zeta^{m(n-1)} \zeta^{n-1} = \zeta^{(m+1)(n-1)}.$ Thus, we have shown that for all $n \geq 3$, $\sigma \zeta^m \sigma = \zeta^{m(n-1)}$ for all $m \geq 1$.
\end{proof}


6. If $S$ is awesome and $x \in S$, then for all $n \geq 1$ the element $x^{3^{n}} \in S$.  (e.g., $x^3$ and $x^9$ and $x^{27}$)

\begin{proof}
    We prove this claim through induction. We first note that $x^1 = x^{3^0} \in S$. We also note that $(x, x) = x x x = x^{3^1} \in S$. Furthermore, we see that $(x^3, x^3) = x^3 x^3 x^3 = x^{3^2} \in S$. We induct on $x^3$ raised to some power, and assume this relation holds true for $m$. We shall prove that this claim is true for the $m+1$ power. Since $x^{3^m}$ is in an awesome set $S$, the element $(x^{3^m}, x^{3^m})$ must also be in $S$. Hence, we know that $(x^{3^m}, x^{3^m}) = x^{3^m} x^{3^m} x^{3^m} = x^{3^{m+1}}$ is an element that lies in $S$. Therefore, $x^{3^{n}}$ will be an element of an awesome set $S$ for all $n \geq 0$.
\end{proof}


6.a (side-quest) Show that $x^5 \in S$.
\begin{proof}
    We prove this directly. Let $S$ be an awesome set and suppose $x \in S$. Given that $S$ is awesome, We know that $x x x = x^3$ must lie in $S$. Furthermore, since $x$ and $x^3$ are elements of $S$, we know then that $(x, x^3) = x (x^3) x = x^5$ must also be in $S$. Thus, $x^5$ is in $S$.
\end{proof}

7. Show that $x^{2n+1} \in S$ for all $n \geq 0$.  (Use induction please)

\begin{proof}
   Let $x \in S$. Because $S$ is an awesome set, we know that $x, x^3, x^7$, and $x^{15}$ must also be elements of $S$. 

   We induct on the exponent of $x$ and assume this relation holds for any odd degree $2n-1$ for all $n \geq 0$. We shall prove this holds for all odd degrees $2n +1$ for all $n \geq 0$. Since  $x^{2n-1}$ and $x$ are elements of $S$, $(x, x^{2n-1}) = x x^{2n-1} x = x^{2n+1}$ must also be in $S$. Thus, $x^{2n+1}$ for all $n \geq 0$ must be in $S$.
   
\end{proof}

8. If $x$ has finite order and its order is an odd number, then show that $1 \in S$.

\begin{proof}
    Let $S \subseteq G$ be an awesome set of a group with a multiplicative law of composition. By the previous inductive proof, we know that if an element of degree $1$ is in an awesome set, all odd powers of said element are also in the awesome set. Thus, let $x \in S$ be an element of $G$ with finite odd order $n$. Consider the element $(x^{n}, x^{n}) = (x^{n})x^{n}(x^{n}) = x^{3n}$ which must be in $S$. Through algebraic manipulation, we see that $x^{3n}= (x^{n})^3 = 1^3.$ Thus, $1 \in S$.
\end{proof}

9. If $x$ has finite order and its order is an odd number, then show that for all $m \geq 0$, we have $x^{2m} \in S$.

\begin{proof}
    Let $x$ be an element of an awesome set $S$ with odd order $n \geq 0$. By the previous constructive proof, we know that if an element of finite, odd order is in an aweomse set, then $1$ must be in the awesome set. Thus, $1$ must be in $S$. Since $x$ and $1$ are in $S$, what must follow is that $(x, 1) = x 1 x = x^2$ is in $S$. 

    We induct on the exponent and assume that this holds true up to $2(m-1)$ where $n \neq m \geq 0$, and we shall prove this is true for degree $2m$. Because we assumed that $x^{2(m-1)} \in S$, we note that $(x, x^{2(m-1)}) = x x^{2(m-1)} x = x^{1 + 2m -2 +1} = x^{2m}$ must also be in $S$. Thus, $x^{2m} \in S$ for all $m \geq 0$.
\end{proof}
     
10. If $x$ has finite order and its order is an odd number, then show that for all $m \geq 0$, we have $x^{m} \in S$ (and $x^{-1} \in S$).*****


11. If $G$ is a group with $|G|$ odd, then every element of $G$ has finite order and that order is odd.*****

\begin{proof}
    Let $G$ be a group of finite odd order. Since $G$ having finite order, let 

    \[|G| = p^{k_1}_{1}\times p^{k_2}_{2} \times \cdots \times p^{k_n}_{n}, \quad \text{where each} \quad p^{k_i}_{i} \quad \text{is a powered-prime.}\] 

\end{proof}

Nudge for 11:  Cauchy's theorem???   Lagrange's theorem

12. Question: if a group has a prime order, does it have to be cyclic?
 
\begin{proof}
    Let $G$ be a group such that $|G| = p$ for some prime $p$. Notice that because $p$ is prime its factorization is $p = 1 p$. Since the order of every element in $G$ must divide the order of the group and the identity element is the only element of order $1$, any other element of $G$ must have order $p$. . Thus, let $a \in G$ be an element of order $p$ that is not the identity. What follows is that $\langle a \rangle$ generates group $G$, and thus $G$ must be cyclic.
\end{proof}

Rework the above proof.  Also why is $\langle a \rangle = G$?  (Hint:  how many elements are in the set $\langle a \rangle$?)

13. Question: if $S$ is an awesome subset of $\mathbb{Z}/15\mathbb{Z}$ and $3 \in S$, then what can we say about $S$?  [What else would have to be in there?  Does $S$ have to be a subgroup?] ****



14. Reading: what's a ``direct product''?  What's a ``semi-direct product''?   Get a few examples of each. (*Read a bit, much is done.)


Bonus: 
15. Is the subset of integers as a ring an awesome set, but a group? Matrices that do that??// what would the ring need to satisfy, both literally and consequently?
 
16. Multiplication is aba, does addition be really nice?? Does it have to be different... Is addition like a subgroup???? Not a subgrin, or ideal, it's kind of something different.

Q. Since $S_3 \cong D_3$, does there exist a nice pattern ("structure") of subsets in the symmetric group like the dihedral group up to the $n$-th case where these subsets are guaranteed awesome... Even though $S_n \cong D_n$ is false?

\newpage

We consider the dihedral group $D_{3} = \langle x, y | x^2 = y^3 = 1 \quad \text{and} \quad yx = xy^2\rangle$.

(*Not sure how one could generalize finding the $aba$-set of a general group $G$, but)

We consider the $aba$-set $S := \{x, xy, xy^2\} \subseteq D_{3}$. We know this set to be $aba$-closed by not only itself, but the entire dihedral group.

Since $S$ is $aba$-closed with the entire group, one idea that comes to mind is that of an analog to a two-sided \textit{coset} of a group---however, recall that $S$ is not a group at all as it lacks the identity element. We will refer to these "cosets" as \textit{semi-cosets}.

We want to be able to recognize things like subgroups in a general group, so we observe what happens when we attempt to redefine these semi-cosets using the maximal $aba$-set $S$ with the elements of $D_{3}$.

We compute the following:

\begin{itemize}
    \item $1 S 1 = 1 S = S 1 = S$
    \item $y S y = y S = S y = S$
    \item $y^2 S y^2 = y^2 S = S y^2 = S$
    \item $x S x = S$; but $xS= Sx = \{1, y, y^2\} \neq S$
    \item $(xy) S (xy) = S$; but $(xy) S= S (xy) = \{1, y, y^2\} \neq S$
    \item $(xy^2) S (xy^2) = S$; but $(xy^2) S= S (xy^2) = \{1, y, y^2\} \neq S$
\end{itemize}

Let $D_{3} / L_{D_3 - L}$ be the set of $aba$-closed semi-cosets with chosen representatives such that they leave the coset unchaged.

More generally speaking,
\[
\{ g\in G : gL_{(aba)} = L_{(aba)}g = L_{(aba)}\}.
\]

Because $L_{(aba)}$ is an $aba$-closed semi-coset with collected group elements. Composition can be defined as

\[
(aL_{(aba)}) (b L_{(aba)}) = 
a (L_{(aba)} b ) L_{(aba)} = 
a (b L_{(aba)}) L_{(aba)} = 
(ab) L_{(aba)}L_{(aba)} = 
(ab) L_{(aba)}. 
\]

Going back to our example with $D_{3} / L_{D_3 - L}$, the set then

\[
\{1, y, y^2\}
\]

The MC-set $L_{aba} = \{x, xy, xy^2\} \subseteq D_{3}$ is maximal via direct construction.
are precisely the elements of $D_3$ such that for each $g \in \{1, y, y^2\}$, this "strong conjugation" by $g$ preserves $S$ in its entirety:
\[
g S g = gS = Sg= S.
\]

We see that the set of such elements forms a subgroup. So, we claim that $D_{3} / L_{D_3 - L} \cong \langle y | y^3 =1 \rangle \cong C_3.$

Lemma. 

\[(\Z / 5\Z \cross \Z / 5\Z ) \rtimes_{\psi} \Z / 3\Z\]
\end{document}
